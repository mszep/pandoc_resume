% Copyright 2013 Christophe-Marie Duquesne <chmd@chmd.fr>
% Copyright 2014 Mark Szepieniec <http://github.com/mszep>
% 
% ConText style for making a resume with pandoc. Inspired by moderncv.
% 
% This CSS document is delivered to you under the CC BY-SA 3.0 License.
% https://creativecommons.org/licenses/by-sa/3.0/deed.en_US

\startmode[*mkii]
  \enableregime[utf-8]  
  \setupcolors[state=start]
\stopmode

\setupcolor[hex]
\definecolor[titlegrey][h=757575]
\definecolor[sectioncolor][h=397249]
\definecolor[rulecolor][h=9cb770]

% Enable hyperlinks
\setupinteraction[state=start, color=sectioncolor]

\setuppapersize [A4][A4]
\setuplayout    [width=middle, height=middle,
                 backspace=20mm, cutspace=0mm,
                 topspace=10mm, bottomspace=20mm,
                 header=0mm, footer=0mm]

%\setuppagenumbering[location={footer,center}]

\setupbodyfont[11pt, helvetica]

\setupwhitespace[medium]

\setupblackrules[width=31mm, color=rulecolor]

\setuphead[chapter]      [style=\tfd]
\setuphead[section]      [style=\tfd\bf, color=titlegrey, align=middle]
\setuphead[subsection]   [style=\tfb\bf, color=sectioncolor, align=right,
                          before={\leavevmode\blackrule\hspace}]
\setuphead[subsubsection][style=\bf]

\setuphead[chapter, section, subsection, subsubsection][number=no]

%\setupdescriptions[width=10mm]

\definedescription
  [description]
  [headstyle=bold, style=normal,
   location=hanging, width=18mm, distance=14mm, margin=0cm]

\setupitemize[autointro, packed]    % prevent orphan list intro
\setupitemize[indentnext=no]

\setupfloat[figure][default={here,nonumber}]
\setupfloat[table][default={here,nonumber}]

\setuptables[textwidth=max, HL=none]

\setupthinrules[width=15em] % width of horizontal rules

\setupdelimitedtext
  [blockquote]
  [before={\setupalign[middle]},
   indentnext=no,
  ]


\starttext

\section[martin-piegay]{Martin Piegay}

\thinrule

\startblockquote
I am a young software engineer working in
\useURL[url1][https://www.zenika.com/][][Zenika]\from[url1] Lyon (FR). I
code in Java, JavaScript and Golang using several frameworks like Spring
or ReactJS \ldots{} I feel concerned about quality of code and I do my
best to improve it. I believe in open-source, I recently contributed the
new reverse proxy
\useURL[url2][https://traefik.io/][][Træfik]\from[url2].
\stopblockquote

\thinrule

\subsection[experience]{Experience}

\startdescription{June 2017 - Today}
  {\bf Backend Software Development on Linkysup at
  \useURL[url3][http://www.enedis.fr/compteur-communicant][][Enedis]\from[url3]};
  \useURL[url4][https://www.zenika.com/][][Zenika]\from[url4] Lyon (FR)
\stopdescription

Developing a software for monitoring and remote controlling 35 millions
of electricity meters installed at private individuals.

\startitemize
\item
  Tech: Java 7, JUnit, AssertJ, Mockito, Spring, ZK, SQL/Oracle,
  Mybatis, Elasticsearch, Drools, Apache Kafka, Apache Flume, WebLogic,
  Maven, Git, Jenkins, Bitbucket, Jira, \ldots{}
\item
  Methodology: Agile Scrum, auto organized team with 6 dev/devops and 5
  functional coworkers, experiencing remote working, part of a
  60-coworkers project.
\stopitemize

\startdescription{December 2016 - June 2017}
  {\bf Fullstack Software Development for Gattefosse};
  \useURL[url5][https://www.zenika.com/][][Zenika]\from[url5] Lyon (FR)
\stopdescription

Developing from scratch a business website which allows clients to
manage medias, users, products and technical documents.

\startitemize
\item
  Tech back : Java 8, Spring 4 (Security, Data JPA, MVC), Hibernate,
  Elasticsearch, JUnit, AssertJ, Mockito, Maven
\item
  Tech front : JavaScript ES6, npm, Webpack, ReactJS with Redux,
  Jest/Jasmine, SASS, flexbox
\item
  CMS : Wordpress with custom plugins
\item
  Tech devops : Git, Gitlab with CI, Apache, MySql, Ansible
\item
  Methodology: Agile Scrum, 3 dev, 1 devops, remote clients
\stopitemize

\startdescription{February 2015 - August 2016}
  {\bf Golang Development on
  \useURL[url6][https://traefik.io/][][Træfik]\from[url6]};
  \useURL[url7][https://containo.us/][][Containous]\from[url7] - Lyon
  (France)
\stopdescription

6-months end-of-courses internship on
\useURL[url8][https://traefik.io/][][Træfik]\from[url8]. Developing 2
packages which provide configuration interfaces for Golang programs:

\startitemize
\item
  \useURL[url9][https://github.com/containous/flaeg][][Flæg]\from[url9]:
  Dynamic Command Line Interface
\item
  \useURL[url10][https://github.com/containous/staert][][Stært]\from[url10]:
  Reads, Merges and Store configurations
\stopitemize

Tech:

\startitemize
\item
  Tech back : Golang 1.7
\item
  Tech devops : Git, Github, Travis CI, Docker, Key/Value Distibuted
  Databases
\item
  Methodology: Open Source Projects, Code Reviews, Pull Requests,
  Startup
\stopitemize

\subsection[education]{Education}

\startdescription{2013-2016}
  {\bf Engineering degree};
  \useURL[url11][https://www.telecom-st-etienne.fr/][][Télécom
  Saint-Étienne]\from[url11] (FR)

  {\em Specialization in Computer, Network Sciences}
\stopdescription

\startdescription{2015-2016}
  {\bf Erasmus student exchange};
  \useURL[url12][https://www.cvut.cz/en][][Czech Technical University in
  Prague]\from[url12] (CZ)

  {\em 1 semester : MSc classes taught in English at the Faculty of
  Information Technologies}
\stopdescription

\subsection[technical-experience]{Technical Experience}

\startdescription{My Cool Side Project}
  For items which don't have a clear time ordering, a definition list
  can be used to have named items.

  \startitemize[packed]
  \item
    These items can also contain lists, but you need to mind the
    indentation levels in the markdown source.
  \item
    Second item.
  \stopitemize
\stopdescription

\startdescription{Open Source}
  List open source contributions here, perhaps placing emphasis on the
  project names, for example the {\bf Linux Kernel}, where you
  implemented multithreading over a long weekend, or {\bf node.js} (with
  \useURL[url13][http://nodejs.org][][link]\from[url13]) which was
  actually totally your idea\ldots{}
\stopdescription

\startdescription{Programming Languages}
  {\bf first-lang:} Here, we have an itemization, where we only want to
  add descriptions to the first few items, but still want to mention
  some others together at the end. A format that works well here is a
  description list where the first few items have their first word
  emphasized, and the last item contains the final few emphasized terms.
  Notice the reasonably nice page break in the pdf version, which
  wouldn't happen if we generated the pdf via html.

  {\bf second-lang:} Description of your experience with second-lang,
  perhaps again including a
  \useURL[url14][https://github.com/githubuser/superlongprojectname][][link]\from[url14],
  this time placing the url reference elsewhere in the document to
  reduce clutter (see source file).

  {\bf obscure-but-impressive-lang:} We both know this one's pushing it.

  Basic knowledge of {\bf C}, {\bf x86 assembly}, {\bf forth},
  {\bf Common Lisp}
\stopdescription

\subsection[extra-section-call-it-whatever-you-want]{Extra Section, Call
it Whatever You Want}

\startitemize
\item
  Human Languages:

  \startitemize[packed]
  \item
    English (native speaker)
  \item
    ???
  \item
    This is what a nested list looks like.
  \stopitemize
\item
  Random tidbit
\item
  Other sort of impressive-sounding thing you did
\stopitemize

\thinrule

\startblockquote
\useURL[url15][mailto:email@example.com][][email@example.com]\from[url15]
• +00 (0)00 000 0000 • XX years old\crlf
address - Mytown, Mycountry
\stopblockquote

\stoptext
